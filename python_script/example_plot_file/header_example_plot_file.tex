% LATEX Dokument für die Diplomarbeit
% Header
\documentclass[pdftex]{scrartcl}
% Seitengröße
\usepackage{typearea}

\usepackage{plotdata_latex}

% \areaset{30cm}{28.5cm}
% \setlength{\paperwidth}{11cm}
% % \setlength{\paperheight}{14.5cm}
% % \areaset{10cm}{14.5cm}
% \setlength{\paperheight}{28.5cm}
% Seitenlayout
% Farbe
\usepackage{color}
% Deutsch
\usepackage[utf8]{inputenc}
% Schrifttyp
\usepackage{mathptmx}
\usepackage[scaled=1.50]{helvet}
\usepackage{courier}
% Absätze
\setlength{\parindent}{0cm} 
\setlength{\parskip}{3mm}
% Grafikpaket
\usepackage{graphicx}
\usepackage[FIGTOPCAP,nooneline, large]{subfig}
% \usepackage{floatflt}
% Mathematsche Formeln
\usepackage{amsmath}
\usepackage{amsthm}
\usepackage{amsfonts}
\usepackage{amssymb}
\usepackage{array}

% Fancybox
\usepackage{fancybox}

% BEFEHLE
\newcommand{\vek}[1]{\underline{#1}}
\newcommand{\mat}[1]{\underline{\underline{#1}}}
\newcommand{\vx}{\vek{x}}
\newcommand{\dvx}{d\vek{x}}
% \newcommand{\grad}{{\rm grad}\,}
% \newcommand{\dif}{{\rm div}\,}
% \newcommand{\rot}{{\rm rot}\,}
\newcommand{\grad}{\nabla}
\newcommand{\dif}{\nabla \cdot}
\newcommand{\rot}{\nabla \times}
\newcommand{\colvek}[3]{
  \left(
    \begin{array}{c}
      #1 \\ #2 \\ #3
    \end{array}
  \right) }
\newcommand{\minkvek}[4]{
  \left(
    \begin{array}{c}
      #1 \\ #2 \\ #3 \\ #4
    \end{array}
  \right) }
\newcommand{\oiint}{\iint}
\newcommand{\es}[1]{\tilde{\vek{e}}_#1}
\newcommand{\eps}{\varepsilon}
\newcommand{\vA}{\vek{A}}
\newcommand{\vD}{\vek{D}}
\newcommand{\vE}{\vek{E}}
\newcommand{\vH}{\vek{H}}
\newcommand{\vF}{\vek{F}}
\newcommand{\vB}{\vek{B}}
\newcommand{\vM}{\vek{M}}
\newcommand{\vK}{\vek{K}}
\newcommand{\vL}{\vek{L}}
\newcommand{\vR}{\vek{R}}
\newcommand{\vS}{\vek{S}}
\newcommand{\vU}{\vek{U}}
\newcommand{\ve}{\vek{e}}
\newcommand{\vj}{\vek{j}}
\newcommand{\vk}{\vek{k}}
\newcommand{\vm}{\vek{m}}
\newcommand{\vn}{\vek{n}}
\newcommand{\vp}{\vek{p}}
\newcommand{\vr}{\vek{r}}
\newcommand{\vu}{\vek{u}}
\newcommand{\vv}{\vek{v}}
\newcommand{\vw}{\vek{w}}
\newcommand{\dtv}{\vek{\dot{v}}}
\newcommand{\dtr}{\vek{\dot{r}}}
\newcommand{\dtx}{\vek{\dot{x}}}
\newcommand{\cph}{c_{\rm Ph}}
\newcommand{\epst}{\mat{\varepsilon}}
\newcommand{\mut}{\mat{\mu}}
\newcommand{\pt}{\partial_t}
\newcommand{\px}{\partial_x}
\newcommand{\pxx}{\partial_{\vx}}
\newcommand{\ppara}{\partial_{\parallel}}
\newcommand{\psenk}{\partial_{\perp}}
\newcommand{\pvv}{\partial_{\vv}}
\newcommand{\pr}{\partial_r}
\newcommand{\ereig}{\mathcal{E}}
\newcommand{\rgam}{\sqrt{1-v^{2}/c^{2}}}
\newcommand{\I}{\rm I}
\newcommand{\II}{\rm II}
\newcommand{\Ls}{{L_{\rm S}}}
\newcommand{\mLs}{{\mat{L}_{\rm S}}}

\renewcommand{\figurename}{Fig.} 

% \newcommand{\length}{7cm}
% \newcommand{\lengthPaper}{7.5cm}
% \newcommand{\heightPaper}{8cm}
% 
% 
% \setlength{\paperwidth}{\lengthPaper}
% \setlength{\paperheight}{\heightPaper}
% \areaset{\length}{\length}


\begin{document}


\begin{figure}
	\centering
	\begin{picture}(0,0)%
% 		\includegraphics[width=7cm]{Enceladus_B0000.png}%
		\input{plotdata_latex.tex}
	\end{picture}%
	\setlength{\unitlength}{7cm}%
	%
	\begingroup\makeatletter\ifx\SetFigFont\undefined%
	\gdef\SetFigFont#1#2#3#4#5{%
	\reset@font\fontsize{#1}{#2pt}%
	\fontfamily{#3}\fontseries{#4}\fontshape{#5}%
	\selectfont}%
	\fi\endgroup%
	\begin{picture}(1,1.02)(0.15,-1.0)
		\put(\xlabelpos,-0.975){\makebox(0,0)[lb]{\smash{{\SetFigFont{16}{16}{\familydefault}{\mddefault}{\updefault}{\color[rgb]{0,0,0}\textbf{\xlabel}}%
		}}}}
		\put(0.15,-0.06){\makebox(0,0)[lb]{\smash{{\SetFigFont{16}{16}{\familydefault}{\mddefault}{\updefault}{\color[rgb]{0,0,0}\textbf{\ylabel}}%
		}}}}
%		\put(\unitxpos,-0.045){\makebox(0,0)[lb]{\smash{{\SetFigFont{16}{16}{\familydefault}{\mddefault}{\updefault}{\color[rgb]{0,0,0}\textbf{\unit}}%
		\put(0.80,-0.045){\makebox(0,0)[lb]{\smash{{\SetFigFont{16}{16}{\familydefault}{\mddefault}{\updefault}{\color[rgb]{0,0,0}\textbf{\unit}}%
		}}}}
	\end{picture}%
\end{figure}


% \section*{Tests for Neutral Drag and Ionization}
% \pagestyle{empty}
% \vspace{1cm}
%  \begin{figure}[!h]
%   \centering
% \vspace{1cm}
% % \vspace{-1.5cm}
% \input{plotdata_latex.tex}
%  \end{figure}
% \begin{tabular}
% % \begin{center}
% % use packages: array
% \begin{tabular}{lll}
% \input{plotdata_latex.tex}
% \end{tabular}
% % \end{center}
% \end{tabular} 

\end{document}
